\section{数集 \textbullet 确界原理}

本节中我们先定义 $\R$ 中两类重要的数集——区间和邻域,然后讨论有界集,并给出确界的定义和确界原理.

\subsection{区间和邻域}

设 $a,b\in \R$ ,且 $a<b$ .我们称数集 $\{x\mid a<x<b\}$ 为 \textbf{开区间} ,记作 $(a,b)$ ; 称数集 $\{x\mid a\le x \le b\}$ 为 \textbf{闭区间} ,记作 $[a,b]$ ;称数集 $\{x\mid a\le x < b\}$ 和 $\{x\mid a< x \le b\}$ 为 \textbf{半开半闭区间} ,分别记作 $(a,b]$ 和 $[a,b)$. 以上这几类区间统称为 \textbf{有限区间}. 从数轴上来看,开区间 $(a,b)$ 表示 $a,b$ 两点间所有点的集合,闭区间 $[a,b]$ 比 开区间 $(a,b)$ 多两个端点,半开半闭区间 $(a,b]$ 和 $[a,b)$ 则比 开区间 $(a,b)$ 多一个端点.

数集 $\{x\mid x\ge a\}$ 记作 $[a,+\infty)$ ,这里符号 $\infty$ 读作 “无穷大”, $+\infty$ 读作 “正无穷大”. 类似地,我们记
\begin{align*}
    &(-\infty,a] = \{x \mid x \le a\} \\
    &(a,+\infty) = \{x \mid x > a\} \\
    &(-\infty,a) = \{x \mid x < a\} \\
    &(-\infty,+\infty) = \{x \mid -\infty < x < +\infty\} = \R
\end{align*}
其中 $-\infty$ 读作“负无穷大”. 以上这几类数集统称为 \textbf{无限区间}.有限区间和无限区间统称为 \textbf{区间}.

设 $a\in \R,\delta>0$ .称集合 $\{x \mid \abs{x-a}<\delta\}$ 为 \textbf{点 $a$ 的 $\delta$ 邻域}, 记作 $U(a,\delta)$,或简记为 $U(a)$ ,即有
\[
U(a,\delta) = \{x \mid \abs{x-a}<\delta\} = (a-\delta,a+\delta)
\]

\textbf{点 $a$ 的空心 $\delta$ 邻域}定义为 
\[
U^\circ(a,\delta) = \{x \mid 0< \abs{x-a}<\delta\} = (a-\delta,a) \cup (a,a+\delta)
\]
它也可以简记为 $U^\circ(a)$. 注意到,$U^\circ(a,\delta)$ 和 $U(a,\delta)$ 的差别在于: $U^\circ(a,\delta)$ 不包含点 $a$.

此外,我们还常用到以下几种邻域:

\textbf{点 $a$ 的 $\delta$ 左邻域} $U_-(a,\delta)=(a-\delta,a]$,简记为 $U_-(a)$.

\textbf{点 $a$ 的 $\delta$ 右邻域} $U_+(a,\delta)=[a,a+\delta)$,简记为 $U_+(a)$.

\textbf{点 $a$ 的空心 $\delta$ 左邻域} $U^\circ_-(a,\delta)=(a-\delta,a)$,简记为 $U^\circ_-(a)$.

\textbf{点 $a$ 的空心 $\delta$ 右邻域} $U^\circ_+(a,\delta)=(a,a+\delta)$,简记为 $U^\circ_+(a)$.

\textbf{$\infty$ 邻域} $U(\infty)=\{x \mid \abs{x}>M\}$ ,其中 $M$ 为充分大的正数(下同).

\textbf{$+\infty$ 邻域} $U(+\infty)=\{x \mid x>M\}$.

\textbf{$-\infty$ 邻域} $U(-\infty)=\{x \mid x<-M\}$.

\subsection{有界集\textbullet 确界原理}

\begin{definition}[有界数集]
    设 $S$ 为 $\R$ 中的一个数集. 
    \begin{itemize}
        \item 若存在数 $M$,使得对一切 $x\in S$, 都有 $x \le M$ ,则称 $S$ 为 \textbf{有上界的数集} ,称 $M$ 为 $S$ 的一个 \textbf{上界}.
        \item 若存在数 $L$,使得对一切 $x\in S$, 都有 $x gle M$ ,则称 $S$ 为 \textbf{有下界的数集} ,称 $L$ 为 $S$ 的一个 \textbf{下界}. 
        \item 若数集 $S$ 既有上界又有下界,则称 $S$ 为\textbf{有界集}.
    \end{itemize}
\end{definition}

\begin{example}[正整数集无上界]
    证明数集 $\N_+ = \{ n \mid n \mbox{为正整数}\}$ 有下界而无上界.
\end{example}

\begin{proof}
    显然,任何一个不大于1的实数都是 $\N_+$ 的下界,故 $\N_+$ 为有下界的数集.

    为证 $\N_+$ 无上界,按照定义只需证明:对于无论多么大的数 $M$, 总存在某个正整数 $n_0 \in \N_+$ ,有 $n_0>M$. 事实上,对任何正数 $M$ (无论多么大),总可取 $n_0=\lfloor M \rfloor +1$ ,则 $n_0 \in \N+$ ,且 $n_0>M$. 这就证明了 $\N_+$ 无上界.
\end{proof}

读者还可以自行证明:任何有限区间都是有界集,无限区间都是无界集; 由有限个数组成的数集是有界集.

若数集 $S$ 有上界,则显然它有无穷多个上界,而其中最小的一个上界常常具有重要的作用,称它为数集 $S$ 的上确界. 同样,有下界数集的最大下界,称为该数集的下确界.下面给出数集的上确界和下确界的精确定义. 

\begin{definition}[上确界]\label{def:sup}
    \renewcommand{\theenumi}{\roman{enumi}}
    \renewcommand{\labelenumi}{\normalfont (\theenumi)}
    设 $S$ 是 $\R$ 中的一个数集.若数 $\eta$ 满足:
    \begin{enumerate}
        \item 对一切 $x\in S$ ,有 $x\le \eta$, 即 $\eta$ 是 $S$ 的上界
        \item 对任何 $\alpha<\eta$ ,存在 $x_0\in S$ ,使得 $x_0> \alpha$, 即 $\eta$ 又是 $S$ 的最小上界
    \end{enumerate}
    则称数 $\eta$ 为数集 $S$ 的\textbf{上确界},记作 $\eta=\sup S$
\end{definition}

\begin{definition}[下确界]\label{def:inf}
    \renewcommand{\theenumi}{\roman{enumi}}
    \renewcommand{\labelenumi}{\normalfont (\theenumi)}
    设 $S$ 是 $\R$ 中的一个数集.若数 $\xi$ 满足:
    \begin{enumerate}
        \item 对一切 $x\in S$ ,有 $x\ge \xi$, 即 $\xi$ 是 $S$ 的下界
        \item 对任何 $\beta>\xi$ ,存在 $x_0\in S$ ,使得 $x_0< \beta$, 即 $\xi$ 又是 $S$ 的最大上界
    \end{enumerate}
    则称数 $\xi$ 为数集 $S$ 的\textbf{下确界},记作 $\xi=\inf S$
\end{definition}

上确界与下确界统称为 \textbf{确界}.

\begin{example}
    设 $S=\{ x \mid x \mbox{为区间} (0,1) \mbox{上的有理数} \}$. 试按照上下确界的定义验证: $\sup S =1,\inf S=0$ .
\end{example}

\begin{proof}
    \renewcommand{\theenumi}{\roman{enumi}}
    \renewcommand{\labelenumi}{\normalfont (\theenumi)}
    先验证 $\sup S=1$:
    \begin{enumerate}
        \item 对一切的 $x\in S$ ,明显有 $x\le 1$ ,即 1 确实是 $S$ 的上界.
        \item 对于任何的 $\alpha <1$, 
        \begin{enumerate}
            \item[\textbullet] 若 $\alpha \le 0$ ,则任取 $x_0 \in S$,都有 $x_0>\alpha$
            \item[\textbullet] 若 $\alpha>0$ ,则由有理数集在实数集中的稠密性可知,在 $(\alpha,1)$ 上必有有理数 $x_0$ , 即存在 $x_0 \in S$,使得 $x_0>\alpha$ .
        \end{enumerate}
    \end{enumerate}

    类似地,可以验证 $\inf S=0$ .
\end{proof}

读者还可自行验证:闭区间 $[0,1]$ 的上、下确界分别为 1 和 0;对于数集 $E=\{\frac{(-1)^n}{n} \mid n=1,2,\cdots\}$ ,有 $\sup E=\frac{1}{2},\inf E =-1$;正整数集 $\N_+$ 有下确界 $\inf \N_+=1$ ,而没有上确界.

\begin{annotation}
    由上(下)确界的定义可见,若数集 $S$ 存在上(下)确界,则一定是唯一的.又若数集 $S$ 存在上、下确界,则有 $\inf S \le \sup S$.
\end{annotation}
\begin{annotation}
    从上面一些例子可见,数集 $S$ 的确界可能属于 $S$ ,也可能不属于 $S$.
\end{annotation}

\begin{example}[确界与最值]
    设数集 $S$ 有上确界.证明
    \[
    \eta = \sup S \in S \iff \eta = \max S
    \]
\end{example}

\begin{proof}
    \renewcommand{\theenumi}{\roman{enumi}}
    \renewcommand{\labelenumi}{\normalfont (\theenumi)}
    \biyao 设 $\eta = \sup S \in S$, 则对一切 $x\in S$ ,都有 $x\le \eta$,而 $\eta\in S$,故 $\eta$ 是数集 $S$ 中最大的数,即 $\eta=\max S$.

    \chongfen 设 $\eta=\max S$,则 $\eta\in S$;下面验证 $\eta = \sup S$:
    \begin{enumerate}
        \item 对一切 $x \in S$ ,均有 $x\le \eta$,即 $\eta$ 是 $S$ 的上界.
        \item 对任何 $\alpha < \eta$ ,只需取 $x_0=\eta\in S$ ,便有 $x_0>\alpha$. 这说明 $\eta=\sup S$.
    \end{enumerate}
\end{proof}

关于数集确界的存在性,我们给出如下确界原理.

\begin{theorem}[确界原理]
    设 $S$ 为非空数集.若 $S$ 有上界,则 $S$ 必有上确界; 若 $S$ 有下界,则 $S$ 必有下确界.
\end{theorem}

\begin{proof}
    \renewcommand{\theenumi}{\roman{enumi}}
    \renewcommand{\labelenumi}{\normalfont (\theenumi)}
    我们只证明关于上确界的结论,后一结论可类似地证明.

    为了叙述方便,不妨设 $S$ 含有非负数.由于 $S$ 有上界,故可以找到非负整数 $n$ ,使得
    \begin{enumerate}
        \item[1)] 对于任何 $x\in S$ ,都有 $x<n+1$.
        \item[2)] 存在 $a_0\in S$ ,使 $a_0\ge n$.
    \end{enumerate}
    对于半开半闭区间 $[n,n+1)$ 作10等分,分点为 $n.1,n.2,\cdots,n.9$ ,则存在 $0,1,2,\cdots,9$ 中的一个数 $n_1$ ,使得 
    \begin{enumerate}
        \item[1)] 对于任何 $x\in S$ ,都有 $x<n.n_1+\frac{1}{10}=n.(n_1+1)$.
        \item[2)] 存在 $a_1\in S$ ,使 $a_1\ge n.n_1$.
    \end{enumerate}.
    对于半开半闭区间 $[n.n_1,n.n_1+\frac{1}{10})$ 作10等分,分点为 $n.n_11,n.n_12,\cdots,n.n_19$ ,则存在 $0,1,2,\cdots,9$ 中的一个数 $n_2$ ,使得 
    \begin{enumerate}
        \item[1)] 对于任何 $x\in S$ ,都有 $x<n.n_1n_2+\frac{1}{10^2}=n.n_1(n_2+1)$.
        \item[2)] 存在 $a_2\in S$ ,使 $a_2\ge n.n_1n_2$.
    \end{enumerate}
    继续不断地10等分在前一步骤中所得到的半开区间,可知对任何 $k=1,2,\cdots$ ,
    存在 $0,1,2,\cdots,9$ 中的一个数 $n_k$ ,使得 
    \begin{enumerate}
        \item[1)] 对于任何 $x\in S$ ,都有 $x<n.n_1n_2\cdots n_k+\frac{1}{10^k}=n.n_1n_2\cdots(n_k+1)$. \hfill (\stepcounter{equation}\theequation)
        \item[2)] 存在 $a_k\in S$ ,使 $a_k\ge n.n_1n_2\cdots n_k$.
    \end{enumerate}
    
    将上述步骤无限地进行下去,得到实数 $\eta=n.n_1n_2\cdots n_k\cdots$.下面证明 $\eta=\sup S$.为此,只需证明:
    \begin{enumerate}
        \item 对一切 $x\in S$ ,有 $x\le \eta$.
        \item 对任何 $\alpha>\eta$ ,存在 $x_0\in S$ ,使得 $x_0> \alpha$
    \end{enumerate}
    倘若结论 (i) 不成立 ,即存在 $x\in S$ ,使 $x>\eta$ ,则可以找到 $x$ 的 $k$ 位不足近似 $x_k$ ,使得
    \[
    x_k > \bar \eta_k = n.n_1n_2\cdots n_k+\frac{1}{10^k}
    \]
    但这与不等式(3) 相矛盾.于是(i)得证.

    现设 $\alpha<\eta$ ,则存在$k$,使 $\eta$ 的 $k$ 位不足近似 $\eta_k>\bar \alpha_k$ ,即 $n.n_1n_2\cdots n_k>\bar \alpha_k$.根据 数$\eta$ 的构造过程,存在 $x_0\in S$ ,有 $x_0 \ge \eta_k > \bar \alpha_k \ge \alpha$. 于是我们得到 $x_0>\alpha$.这说明 (ii) 成立.
\end{proof}

在本书中,确界原理是极限理论的基础,读者应给予充分的重视.

\begin{example}
    设 $A,B$ 为非空数集,满足:对一切 $x\in A$ 和 $y\in B$ 有 $x\le y$.证明:数集 $A$ 有上确界,数集 $B$ 有下确界, 且 \begin{equation}
        \sup A\le \inf B 
    \end{equation} 
\end{example}

\begin{proof}
    由题设可得,数集 $B$ 中任何一个数 $y$ 都是数集 $A$ 的上界, $A$ 中任何一个数 $x$.都是 $B$ 的下界,故由确界原理推知,数集 $A$ 有上确界,数集 $B$ 有下确界.

    现证不等式(4).对任何 $y\in B$ ,$y$ 都是数集 $A$ 的一个上界,而由上确界的定义得,$\sup A$ 是数集 $A$ 的最小上界.因此 $\sup A\le y$. 而此式又表明数 $\sup A$ 是数集 $B$ 的一个下界,故由下确界的定义得 $\sup A\le \inf B$.
\end{proof}

\begin{example}
    设 $A,B$ 为非空有界数集, $S=A\cup B$. 证明:
    \begin{enumerate}
        \item[(i)] $\sup S=\max \{\sup A,\sup B\}$
        \item[(ii)] $\inf S=\min \{\inf A,\inf B\}$
    \end{enumerate}
\end{example}

\begin{proof}
    由于 $S=A\cup B$,易知 $S$ 也是非空有界数集,因此 $S$ 的上、下确界都存在.

    (i) 对于任何的 $x\in S$,有 $x\in A$ 或 $x\in B \implies x\le \sup A$ 或 $x \le \sup B$ ,从而有 $x\le \max \{\sup A,\sup B\}$ ,故得 $\sup S \le \max \{\sup A,\sup B\}$.

    另一方面,对任何的 $x\in A$ ,有 $x\in S \implies x \le \sup S \implies \sup A\le \sup S$; 同理有 $\sup B \le \sup S$.所以有 $\sup S\ge \max \{\sup A,\sup B\}$.

    综上,即可证得 $\sup S\ge \max \{\sup A,\sup B\}$.

    (ii) 可类似证明.
\end{proof}

若把 $+\infty$ 和 $-\infty$ 补充到实数集中,并规定任何一个实数 $a$ 与 $+\infty$ , $-\infty$ 的大小关系为: $-\infty<a<+\infty$,则确界的概念可以扩充为:若数集 $S$ 无上界,则定义 $+\infty$ 为 $S$ 的\textbf{非正常上确界},记作 $\sup S= +\infty$; 若 $S$ 无下界,定义 $-\infty$ 为 $S$ 的\textbf{非正常下确界},记作 $\inf S = -\infty$. 相应地,前面\defref{def:sup},\defref{def:inf}中所定义的确界分别称为 \textbf{正常上、下确界}.

在上述扩充意义下,我们有
\begin{theorem}[推广的确界原理]
    任何一个非空数集必有上、下确界(正常的或非正常的).
\end{theorem}

例如,对于正整数集 $\N_+$ ,有 $\inf \N_+ = 1,\sup \N_+=+\infty$; 对于数集 $S=\{y\mid y=2-x^2,x\in \R\}$,有 $\inf S= -\infty,\sup S=2$.

\homework

\begin{practice}
    用区间表示下列不等式的解:

    (1) $\abs{1-x}-x\ge 0$ \qquad (2)$\abs{x+\frac{1}{x}}\le 6$ 

    (3) $(x-a)(x-b)(x-c)>0$ ($a,b,c$ 为常数,且 $a<b<c$) \qquad (4) $\sin x\ge \frac{\sqrt{2}}{2}$

\end{practice}

\begin{solve}
    (1) $\abs{1-x}-x\ge 0 \iff \abs{1-x}\ge x\iff 1-x\ge x \mbox{或}x-1\ge x \iff x\le \frac{1}{2}$

    用区间表示为 $(-\infty,\frac{1}{2}]$.

    (2) $\abs{x+\frac{1}{x}}\le 6 \iff (x+\frac{1}{x})^2 \le 36 \iff (x^2+1)^2 \le 36x^2 \iff x^4 -34x^2 +1\le 0 $
    
    $\iff 17-12\sqrt{2} \le x^2 \le 17+12\sqrt{2} \iff -3-2\sqrt{2} \le x \le -3+2\sqrt{2} \mbox{或} 3-2\sqrt{2} \le x \le 3+2\sqrt{2}$

    用区间表示为 $[-3-2\sqrt{2},-3+2\sqrt{2}] \cup [3-2\sqrt{2},3+2\sqrt{2}]$.

    (3) 解集为 $\{ x\mid a<x<b \mbox{或} x>c\}$. 写成区间为  $(a,b)\cup (c,+\infty)$.

    (4) $\bigcup\limits_{k\in \mathbb{Z}}[\frac{\pi}{4}+2k\pi,\frac{3\pi}{4}+2k\pi]$
\end{solve}

\begin{practice}
    设 $S$ 为非空数集.试对下列概念给出定义:

    (1) $S$ 无上界 \qquad (2) $S$ 无界
\end{practice}

\begin{solve}
    \hspace{0.5em}(1) $S$ 为非空数集,若对任意的正数 $M$,存在 $x_0\in S$, 使得 $x_0 > M$ ,则称 $S$ 无上界.

    (2) $S$ 为非空数集,若对任意的正数 $M$,存在 $x_0\in S$, 使得 $\abs{x_0} > M$ ,则称 $S$ 无界.
\end{solve}

\begin{practice}
    试证明 数集 $S=\{y\mid y=2-x^2,x\in \R\}$ 有上界而无下界.
\end{practice}

\begin{proof}
    $y=2-x^2\le 2$,故 $S$ 有上界 $2$.

     对于任何一个正数 $M$,都存在 $x_0=\sqrt{3+M}$,使得 $y_0=2-x_0^2=-M-1<-M$.故 $S$无下界.
\end{proof}

\begin{practice}
    求下列数集的上、下确界,并依据定义加以验证:

    (1) $S=\{x\mid x^2<2\}$ \qquad (2)$S=\{x \mid x=n! ,n\in \N_+ \}$

    (3) $S=\{ x\mid x\mbox{为} (0,1) \mbox{上的无理数} \}$ \qquad (4) $S=\{ x \mid x=1-\frac{1}{2^n},n\in \N_+\}$
\end{practice}

\begin{solve}
    (1) $\sup S = \sqrt{2},\inf S = -\sqrt{2}$ . 
    易得,$\sqrt{2}$ 是 $S$ 的一个上界. 对任意的 $\eta<\sqrt{2}$, 若 $\eta<-\sqrt{2}$ ,则 $\forall x\in S$ ,都有 $x>\eta $; 若 $\eta \ge -\sqrt{2}$ , 则存在 $x_0 = \frac{\sqrt{2}+\eta}{2} \in S$ , 有 $x_0>\eta $.由定义得, $\sqrt{2}$ 是 $S$ 的上确界. 同理可验证 $-\sqrt{2}$ 是 $S$ 的下确界.

    (2) $\sup S = +\infty,\inf S = 1$ . 对任意的正数 $M$ ,都存在 $x_0 = \lceil M \rceil ! \in S$, 有 $x_0>M$. 故 $\sup S = +\infty$. 易知, 1 为 $S$ 的一个下界. 对任意的 $ \xi> 1$, 存在 $x_0=1\in S$ , 有 $x_0<\xi$.故 1 为 $S$ 的下确界.

    (3) $\sup S = 1,\inf S = 0$  . 易知 1 为$S$ 的一个上界. 对任意的 $\eta < 1$, 根据无理数的稠密性,在 $(\eta,1)$ 上必然存在无理数 $x_0\in S$,即有 $x_0>\eta$.故 1 为 $S$ 的上确界. 同理可证 $0$ 为 $S$ 的下确界.

    (4) $\sup S = 1,\inf S = \frac{1}{2}$. 对任意的 $x\in S$,有 $x=1-\frac{1}{2^n}\le 1$. 故 1 是 $S$ 的上界. 对任何的 $\eta <1$, 若 $\eta<\frac{1}{2}$,则存在 $x_0=1-\frac{1}{2}=\frac{1}{2} \in S$,有 $x_0>\eta$ ;若 $\eta\ge \frac{1}{2}$, 取 $n=\lceil log_2(\frac{1}{1 - \eta}) \rceil$ , 则存在 $x_0 = 1-\frac{1}{2^n} \in S$ ,有 $x_0>\eta$ . 故 1 是 $S$ 的上确界. 容易验证,下确界即为 $S$ 中的数的最小值 $\frac{1}{2}$.
\end{solve}

\begin{practice}
    设 $S$ 为非空有下界的数集,证明:$\inf S=\xi \in S \iff \xi =\min S$.
\end{practice}
\begin{proof}
    \renewcommand{\theenumi}{\roman{enumi}}
    \renewcommand{\labelenumi}{\normalfont (\theenumi)}
    \biyao 设 $\xi = \inf S \in S$, 则对一切 $x\in S$ ,都有 $x\ge \xi$,而 $\xi \in S$,故 $\xi$ 是数集 $S$ 中最小的数,即 $\eta=\min S$.

    \chongfen 设 $\xi=\min S$,则 $\xi\in S$;下面验证 $\xi = \inf S$:
    \begin{enumerate}
        \item 对一切 $x \in S$ ,均有 $x\ge x$,即 $\xi$ 是 $S$ 的下界.
        \item 对任何 $\beta > \xi$ ,只需取 $x_0=\xi\in S$ ,便有 $x_0< \beta$. 这说明 $\xi=\inf S$.
    \end{enumerate}
\end{proof}

\begin{practice}
    设 $S$ 为非空数集,定义 $S^-=\{ x\mid -x \in S\}$.证明:

    (1) $\inf S^-=-\sup S$ \qquad (2) $\sup S^- = -\inf S$.
\end{practice}

\begin{proof}
    (1) 对任意的 $x\in S^-$ ,都有 $-x\in S$,则 $-x\le \sup S$,故 $x\ge -\sup S$,即 $-\sup S$ 是 $S^-$ 的下界. 对任意的 $\beta>-\sup S$ , 也即 $-\beta < \sup S$ ,都存在 $-x_0\in S$ ,有 $-x_0 > -\beta$ ,则 $x_0\in S^-$ ,且 $x_0 < \beta$. 故 $-\sup S$ 是 $S^-$ 的下确界.

    (2) 对任意的 $x\in S^-$ ,都有 $-x\in S$,则 $-x\ge \inf S$,故 $x\le -\inf S$,即 $-\inf S$ 是 $S^-$ 的上界. 对任意的 $\alpha<-\inf S$ , 也即 $-\alpha > \inf S$ ,都存在 $-x_0\in S$ ,有 $-x_0 < -\alpha$ ,则 $x_0\in S^-$ ,且 $x_0 > \alpha$. 故 $-\inf S$ 是 $S^-$ 的上确界.
\end{proof}

\begin{practice}
    设 $A,B$ 均为非空有界数集,定义数集 $A+B=\{z\mid z=x+y,x\in A,y\in B\}$.

    证明:(1) $\sup (A+B) = \sup A+\sup B$ 
    \qquad (2)$\inf (A+B) = \inf A+\inf B$ 
\end{practice}

\begin{proof}
    (1) \biyao 对任意的 $z\in A+B$ ,有 $z=x+y \le \sup A + \sup B $, 其中 $x\in A,y\in B$. 故 $\sup A + \sup B$ 是 $A+B$ 的一个上界, 则 $\sup (A+B)\le \sup A + \sup B$.
    
    \chongfen 
    对于任意的 $\varepsilon>0$, 存在 $x_0\in A$ , 有 $x_0>\sup A - \frac{1}{2}\varepsilon$ , 存在 $y_0\in B$ , 有 $y_0>\sup B - \frac{1}{2}\varepsilon$. 故 $\sup(A+B) \ge z_0 = x_0+y_0>\sup A+\sup B -\varepsilon$ . 由 $\varepsilon$ 的任意性可知, $\sup(A+B) \ge \sup A+\sup B $.
    
    综上,$\sup(A+B) = \sup A+\sup B $.
    
    (2) \biyao 对任意的 $z\in A+B$ ,有 $z=x+y \ge \inf A + \inf B $, 其中 $x\in A,y\in B$. 故 $\inf A + \inf B$ 是 $A+B$ 的一个下界, 则 $\inf (A+B)\ge \inf A + \inf B$.
    
    \chongfen 
    对于任意的 $\varepsilon>0$, 存在 $x_0\in A$ , 有 $x_0<\inf A + \frac{1}{2}\varepsilon$ , 存在 $y_0\in B$ , 有 $y_0<\inf B + \frac{1}{2}\varepsilon$. 故 $\inf(A+B) \le z_0 = x_0+y_0<\inf A+\inf B +\varepsilon$ . 由 $\varepsilon$ 的任意性可知, $\inf(A+B) \le \inf A+\inf B $.
    
    综上,$\inf(A+B) = \inf A+\inf B $.
\end{proof}

\newsection